\documentclass[journal ]{new-aiaa}
\usepackage[utf8]{inputenc}
\usepackage{textcomp}
\usepackage{subcaption}
\usepackage{float}

\usepackage[euler]{textgreek}

\usepackage{graphicx}
\usepackage{amsmath}
\usepackage[version=4]{mhchem}
\usepackage{siunitx}
\usepackage{longtable,tabularx}
\setlength\LTleft{0pt} 

\newcounter{ctab}
\setcounter{ctab}{1}

\title{Final Report Feedback}

\author{Joseph Spencer \footnote{Undergraduate Researcher, Brigham Young University FLOW Lab.}}
\affil{Brigham Young University, Provo, Utah, 84601} 

\begin{document}

\maketitle


\section{Introduction}

I sent my final report to both Judd Mehr and Adam Cardoza. They each gave me feedback, which I used to update my final report for presentation. \href{Diff.pdf}{Diff.pdf} shows the significant updates I made after receiving feedback. In general, the final document has much more content and better flow than the \href{Draft_1.pdf}{first} and \href{Draft_2.pdf}{second} drafts. The \href{Final_Report.pdf}{final report} is available through this link.

Because many of Adam's and Judd's comments were similar, I should clarify that I have labeled changes from the first draft to the second draft "J" and changes from the second draft to the final report "A."

\section{Judd}

Judd was the first to give me feedback on this report. I made most of the changes between the first and second drafts and added much of the content to the final submission because of his comments. A summary is listed below.

\begin{itemize}
    \item Abstract - Add more material to the abstract including motivation, relation to readers, and a brief summary of the report, research, findings, conclusions, and perspectives.
    \item Nomenclature - Remove "=" signs and adjust units in nomenclature.
    \item Introduction - Add significant content to introduction, splitting it into three paragraphs for motivation, context, and outline.
    \item Procedure - Clarify explanation of Julia language and Xfoil.jl package. Clarify which package does which thing. Also clarify use of the $n=1.1$ safety factor. Make sure you use clear and consistent terminology throughout report.
    \item Optimization - Be sure to use consistent terminology for which optimizer was used. 
    \item Methodology - Include content from project 1 and project 2. Include more math and figures in the methodology section so someone unfamiliar with the project could produce the same results.
    \item Results - Clarify discussion of airfoil identification remaining the same. Add and discuss a thrust constraint.
    \item Discussion - Clarify language again! Also explain why you expected different results. Move process discussion to procedure section.
    \item Conclusion - Write more stuff in the conclusion
    \item Miscellaneous - Include more content from projects 1 and 2. Use consistent names for variables and a consistent citation style, and check grammar before final submission. Change from passive to active voice.
\end{itemize}

\section{Adam}

Adam checked my second draft pdf. It had some updates to the beginning sections from Judd's comments. Some of his additional feedback was similar to Judd's, and some was slightly different. Feedback from both Adam and Judd was considered.

\begin{itemize}
	\item Abstract - Change title and author name. Update wordings and terminology to be more descriptive and less repetitive.
	\item Nomenclature - Remove nomenclature table, because these terms are well-known.
	\item Introduction - Remove paragraph labels added in second draft. Clarify wording about objective function and add references to statements made about Julia.
	\item Methods - Change multiple sections to only methods and results. Add discussion of what different Julia files in the project are.
	\item Methods - Explain why the safety factor $n=1.1$ was used. Explain the optimization function and clarify the inputs, or include it in the optimization table. Correct citation style. 
	\item Results - Make sure it is clear that the optimizer could never change the airfoil or rotor. Discuss why these results were obtained and what they mean. Discuss rotor solidity in more depth.
	\item Results - Clarify wording to distinguish between angle of attack, twist, angle of rotation, and inflow angle.
	\item Conclusion - Discuss findings much more in-depth. Also discuss their meanint.
	\item Miscellaneous - Fix capitalization in figure and table labels.
\end{itemize}

\bibliography{Final_Report}

\end{document}
