\documentclass{article}

\usepackage{graphicx}
\usepackage{amsmath,amsfonts,amssymb}
\usepackage[colorlinks,bookmarks,bookmarksnumbered,allcolors=blue]{hyperref}
\usepackage[capitalise]{cleveref}
\usepackage[top=0.75in]{geometry}
\usepackage[dvipsnames]{xcolor}
\usepackage{amsmath} 
\usepackage{esvect}

\usepackage[utf8]{inputenc}
\usepackage{glossaries}

\makeglossaries

\newglossaryentry{glsy}
{
    	name = Airfoil Thickness,
    	description = {The thingy}
}

\newglossaryentry{latex}
{
    	name = Airfoil Thickness,
    	description = {The thingy}
}

\begin{document}

\author{Joe Spencer}
\title{Airfoil Analysis}
\date{August 27, 2022}
\maketitle

\subsubsection*{Methods}
Pairs of vortex rings can move together along a line, appearing to play leapfrog with each other. This phenomenon can occur when the circulation of one vortex ring pulls a second vortex ring forward, and once the second vortex ring has its own inertia moving it forward it begins to pull the first vortex ring. For this assignment, a set of 4 points marking the top and bottom of 2 vortex rings was modeled to leapfrog off of each other, and different graphs and GIFs were created of this behavior. The code used to solve this problem, as well as output data and graphs from the nine drafts of this project, can be found in the LeapFrog branch (\emph{JoeSpencer1-LeapFrog}) of this \href{https://github.com/JoeSpencer1/497R-Projects.git}{GitHub Repository}. \newline

\gls{glsy} and \gls{latex}.

I used the Julia programming language to model these leapfrogging vortices. Julia was especially handy for this project because it could perform calculations on large matrices quickly. Another feature of Julia that became especially handy for this project is its ability to publish figures easily for help visualizing data. Initially, data was written to a file, but recording data to a file was later skipped in the final version to instead dedicate runtime to examining different initial conditions.\newline

Vortices can be described in vector form by Equation 1. This equation shows that the velocity at a point around a vortex is proportional to its circulation, $\vec{\Gamma}$, divided by its distance from the center, $\vec{r}$. $\vec{\Gamma}$ and $\vec{r}$ are crossed to obtain $\vec{V}$, meaning that if $\vec{\Gamma}$ is in the y-direction and $\vec{r}$ is in the x-direction, then the velocity induced by the vortex will be in the y-direction. This also means that an initial velocity is not required for leapfrogging vortices, since the circulation itself will induce that. For my completion of this lab, I used a cross-section of the vortex rings and the x and z-axes with $\vec{\Gamma}$ entirely on the y-axis, pointing into or out of the page.  \newline

\begin{equation} \label{eq:1}
    \begin{aligned}
        \vec{V} &= \frac{\vec{\Gamma} \times \vec{r}}{2 \pi r^2}
    \end{aligned}
\end{equation}

Although $\vec{\Gamma}$ and $\vec{r}$ are crossed, the denominator of Equation 1 contains an r$^2$ term, which means that circulation has a reciprocal relationship with distance, as also expected intuitively. So, in a case where the vortices are further apart, they will be induce less motion in each other. If the vorticity is increased, the vortices will be induced more strongly and move more quickly in their leapfrogging pattern. \newline

This project allowed some freedom to choose initial conditions. In the final version of the project, I have tried 9 different sets of initial conditions. Each vortex was centered around the x-axis. In Table 1, the starting conditions for each set of leapfrogging vortices are listed. This variety of initial conditions yielded a variety of different results, which will be discussed. Table 1 shows the initial position and velocity coordinates and the initial vorticity for each set of leapfrogging vortices. As shown in the table, differences in each of these initial conditions were tested. \newline

\begin{table}
	\centering
	\title{Table 1: Initial Position, Velocity, and Circulation of Each Vortex \newline}
	\title{\emph{A link to a GIF for each vortex is in its far left column.}}
	\begin{tabular}{ | c | c | c | c | c | c | c | c |}
		\hline
		GIF & $f_m$ (Hz) & 1$^{st}$ [$x_0$,$y_0$,$z_0$] & 1$^{st}$ [$\dot{x_0}$,$\dot{y_0}$,$\dot{z_0}$] & $\Gamma_1$ & 2$^{nd}$ [$x_0$,$y_0$,$z_0$] & 2$^{nd}$ [$\dot{x_0}$,$\dot{y_0}$,$\dot{z_0}$] & $\Gamma_2$ \\ \hline
		\href{https://imgur.com/U4cND6X}{1} & 10 & [1.0,0,1.0] & [0,0,0] & 1.0 & [-1.0,0,1.0] & [0,0,0] & 1.0 \\ \hline
		\href{https://imgur.com/lviamWB}{2} & 10 & [2.0,0,2.0] & [0,0,0] & 1.0 & [-2.0,0,2.0] & [0,0,0] & 1.0 \\ \hline
		\href{https://imgur.com/6jT1mT0}{3} & 100 & [1.0,0,1.0] & [0,0,0] & 1.0 & [-1.0,0,1.0] & [0,0,0] & 1.0 \\ \hline
		\href{https://imgur.com/PtazOFB}{4} & 1 & [1.0,0,1.0] & [0,0,0] & 1.0 & [-1.0,0,1.0] & [0,0,0] & 1.0 \\ \hline	
		\href{https://imgur.com/wSpWEQV}{5} & 10 & [1.0,0,1.0] & [0,0,0] & 1.0 & [-1.0,0,1.0] & [0,0,0] & -1.0 \\ \hline
		\href{https://imgur.com/aZzMnJV}{6} & 10 & [1.0,0,1.0] & [0,0,0] & 2.0 & [-1.0,0,1.0] & [0,0,0] & 1.0 \\ \hline
		\href{https://imgur.com/nKVGuJ9}{7} & 10 & [0.75,0,0.75] & [0.2,0,0] & 2.5 & [-0.8,0,0.6] & [0,0,0] & 2.5 \\ \hline
		\href{https://imgur.com/UZyBsYO}{8} & 10 & [1.0,0,1.0] & [0.006,0,0.008] & 1.0 & [-1.0,0,1.0] & [0,0,0] & 1.0 \\ \hline
		\href{https://imgur.com/tctJPyS}{9} & 10 & [1.0,0,1.0] & [0.2,0,0.1] & 3.0 & [-1.0,0,1.0] & [0.2,0,0.1] & 3.0 \\ \hline
	\end{tabular}
\end{table}

The final program for modeling leapfrogging vortices also allowed the user to adjust the number of steps the program would simulate and the length of each time step, but these parameters were left at 0.1 seconds and 1,000 steps for most of the program. Shorter and longer time steps between measurements were tried, but the lower frequency shown in \href{https://imgur.com/PtazOFB}{GIF 8} diverged from the actual solution for the leapfrogging vortices, and the higher calculation frequency modeled in \href{https://imgur.com/6jT1mT0}{GIF 9} only slowed down the program while measuring fewer vortices that were not substantially different. \newline

Once I understood that I was to model the top and bottom points of 2 vortices, it was relatively smooth sailing to establish a working code. To make this model I used the DeliniatedFiles, Printf, and Plots libraries in Julia. After I had created a good model, I decided to animate it. I eventually found that to be fairly simple. I used the \textcolor{Dandelion}{@animate} function to make a gif for each vortex. These can be viewed either by clicking the colored number links in the table, or by going to the LeapFrog branch of the \href{https://github.com/JoeSpencer1/497R-Projects.git}{GitHub Repository} mentioned previously. \newline

\subsubsection*{Results and Discussion}

This lab examined how vortices can perpetuate each other. The following nine figures show the different paths followed by each vortex pair in this experiment. The results shown in these graphs provide insights about when the phenomenon of vortex leapfrogging occurs, and when it does not.\newline

One interesting relationship to note in the vortex rings that this experiment found is that for vortex rings that go on the longest, the velocity is inversely proportional to the distance. This can be recalled from Equation \ref{eq:1}, which showed that the magnitude of velocity is directly proportional to circulation and ultimately inversely proportional to distance. The 
\href{https://imgur.com/U4cND6X,}{videos} included in this report show that show that for the vortices that continued longest, the highest velocity occurred when \emph{r} was smallest, and lowest velocity came when \emph{r} between the top and bottom vortices of the ring was greatest. \newline

\clearpage

\printglossaries

\end{document}