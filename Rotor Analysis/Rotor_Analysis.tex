\documentclass{article}

\usepackage{graphicx}
\usepackage{amsmath,amsfonts,amssymb}
\usepackage[colorlinks,bookmarks,bookmarksnumbered,allcolors=blue]{hyperref}
\usepackage[capitalise]{cleveref}
\usepackage[top=0.75in]{geometry}
\usepackage[dvipsnames]{xcolor}
\usepackage{amsmath} 
\usepackage{esvect}
\usepackage{hyperref}
\usepackage{graphicx}
\usepackage{subcaption}
\usepackage{stfloats}
\usepackage{array}
\usepackage{booktabs}
\usepackage{amsmath}

\begin{document}

\author{Joe Spencer}
\title{Rotor Analysis}
\date{September 27, 2022}
\maketitle

\subsubsection*{Methods}

\subsubsection*{Results and Discussion}

The \hyperlink{BEM}{Blade Element Moment Theory}

Axial induction factor $a$ Tangential induction factor $a'$

\clearpage

\section{Glossary}
\begin{itemize}
	
	\item \hypertarget{J}{Advance Ratio, $J$} - A rotor's Advance Ratio is a non-dimensional term. It describes the ratio how quickly a rotor is moving relative to the fluid flowing past it. A high advance ratio signifies that either the fluid is moving quickly or the rotor is moving slowly. It is described by the following equation, in which $V_{a}$ is the free stream fluid velocity, $n$ is the rotational velocity, and $D$ is the rotor diameter.
	\begin{equation}
	\begin{aligned}
		J = \frac{V_{a}}{n D}
	\end{aligned}
	\end{equation}
	
	\item \hypertarget{phi}{Angle of Rotation, $\phi$} - The Angle of Rotation, sometimes denoted by the Greek letter $\phi$, is the angle between the freest stream velocity and the velocity of the airfoil as it rotates. It is used in \hyperlink{BEM}{Blade Element Moment Theory} calculations.

	\item \hypertarget{a}{Axial Induction Factor, $a$} - The Axial Induction Factor is the ratio of the reduction in air velocity at an airfoil to its free stream velocity.
	
	\item \hypertarget{BEM}{Blade Element Moment Theory} - The theory used to calculate local forces on a propellor or wind turbine blade. It employs both \hyperlink{BET}{Blade Element Theory} and \hyperlink{MT}{Momentum Theory}. These equations are used to recursively find the \hyperlink{a}{Axial Induction Factor, $a$}, \hyperlink{a'}{Tangential Induction Factor}, and \hyperlink{phi}{Angle of Rotation, $\phi$}
	\begin{equation}
	\begin{aligned}
		\frac{1}{2} W^{2} N c C_{y} = 4 \pi U_{\infty} (1 - a) \times \Omega a' r^{2} \\
		\frac{1}{2} \rho W^{2} N c C_{x} = 4 \pi \rho [(a' \Omega r)^{2} + \Omega^{2}_{\infty} a (1 - a)] r \\
		\sin \phi = \frac{U_{\infty}}{W} (1 - a)
	\end{aligned}
	\end{equation}
In these equations, $a$, $a'$, and $\phi$ are the previously mentioned axial and tangential induction factors and angle of rotation. The airfoil's apparent speed is represented by the letter $W$, $N$ is the number of propellers, $\rho$ is the fluid density, $c$ is the chord length, $C_{x}$ and $C_{y}$ are obtained by the equation below, $U_{\infty}$ is the fluid free velocity, $\Omega$ is the blade's angular speed, and $r$ is the radius to the tip of the blade.
	\begin{equation}
	\begin{aligned}
		C_{x} = c_{l} \cos{\phi} + c_{d} \sin{\phi} \\
		C_{y} = c_{l} \sin{\phi} + c_{d} \cos{\phi}
	\end{aligned}
	\end{equation}
	
	\item \hypertarget{BET}{Blade Element Theory} - Blade Element Theory calculates the forces on a turbine blade by dividing it into finite pieces and summing the forces on all of these pieces. This theory determines the induced velocity and efficiency of a point along a blade using these equations:
	\begin{equation}
	\begin{aligned}
		v_{i} = \sqrt{\frac{T}{A} \frac{1}{2 \rho}} \\
        		\eta = \frac{\tan{\phi}}{\tan{(\phi + \gamma)}}
	\end{aligned}
	\end{equation}
In these equations, $v_{i}$ is the uniform induced velocity across the disk, $T$ the thrust it experiences, $A$ is its area, $\rho$ is the air density, $\phi$ is the angle to the airfoil's plane of rotation as it moves forward, and $\gamma$ is the difference between $\phi$ and $\beta$, what the airfoil's actual angle of rotation would be if it were stationary.

	\item \hypertarget{c}{Chord Distribution} - The Chord Distribution
	
	\item \hypertarget{CP}{Coefficient of Power, $c_{P}$} - A propellor's Coefficient of Power signifies how efficient a wind turbine is. It is the ratio of the power generated by a wind turbine to the total power of the wind flowing through it. The power generated or absorbed by an airfoil can be described by the following equation, where $P$ is power, $c_{P}$ is the coefficient of power, $\rho$ is fluid density, $n$ is the velocity in revolutions per second, and $D$ is the propellor diameter.
	\begin{equation}
	\begin{aligned}
		P = \rho n^{3} D^{5} c_{P}
	\end{aligned}
	\end{equation}
	
	\item \hypertarget{CT}{Coefficient of Thrust, $C_{T}$} - A rotor's Thrust Coefficient determines how much thrust in the "forward" direction an airfoil experiences. Thrust force is directly opposite drag. Please note the similarities and differences between the thrust equation and the \hyperlink{CP}{power equation}.
	\begin{equation}
	\begin{aligned}
		T = \rho n^{2} D^{4} c_{T}
	\end{aligned}
	\end{equation}
	
	\item \hypertarget{eta}{Efficiency, $\eta$} - The Efficiency of a rotor can be described by the following equation, in which $J$ is the rotor's \hyperlink{J}{Advance Ratio}, $c_{T}$ is its \hyperlink{CT}{thrust coefficient}, and $c_{P}$ its \hyperlink{CP}{power coefficient}:
	\begin{equation}
	\begin{aligned}
		\eta = J \frac{c_{T}}{c_{P}}
	\end{aligned}
	\end{equation}
	
	\item \hypertarget{D/D}{Hub-to-Tip Ratio} - The Hub-to-Tip Ratio
	
	\item \hypertarget{MT}{Momentum Theory} - Momentum Theory defines the power required to produce sufficient thrust to maintain momentum in a blade by the following equation, where $T$ is thrust, $\rho$ is density, $A$ is disc area, and $P$ is power:
	\begin{equation}
	\begin{aligned}
        		P = \sqrt{\frac{T^{3}}{2 \rho A}}
	\end{aligned}
	\end{equation}
	
	\item \hypertarget{sigma}{Rotor Solidity, $\sigma$} - Rotor Solidity
	
	\item \hypertarget{a'}{Tangential Induction Factor, $a'$} - The Tangential Induction Factor is the ratio of the increase in air velocity tangential to the airfoil to its free stream velocity.
	
	\item \hypertarget{lambda}{Tip Speed Ratio, $\lambda$} - A wind turbine's Tip Speed Ratio is the inverse of its \hyperlink{J}{Advance Ratio, $J$}. It represents the ratio of the speed of the tip of a turbine blade, or $\omega R$, to the wind speed, $v$.
	\begin{equation}
	\begin{aligned}
		\lambda = \frac{\omega R}{v}
	\end{aligned}
	\end{equation}
	
	\item \hypertarget{T}{Twist Distribution} - The Twist Distribution
	
\end{itemize}

\end{document}