\documentclass{article}

\usepackage{graphicx}
\usepackage{amsmath,amsfonts,amssymb}
\usepackage[colorlinks,bookmarks,bookmarksnumbered,allcolors=blue]{hyperref}
\usepackage[capitalise]{cleveref}
\usepackage[top=0.75in]{geometry}
\usepackage[dvipsnames]{xcolor}
\usepackage{amsmath} 
\usepackage{esvect}
\usepackage{hyperref}
\usepackage{graphicx}
\usepackage{subcaption}
\usepackage{stfloats}
\usepackage{array}
\usepackage{booktabs}
\usepackage{amsmath}

\begin{document}

\author{Joe Spencer}
\title{Rotor Analysis}
\date{September 27, 2022}
\maketitle

\subsubsection*{Methods}

\subsubsection*{Results and Discussion}

The \hyperlink{BEM}{Blade Element Moment Theory}

Axial induction factor $a$ Tangential induction factor $a'$

\clearpage

\section{Glossary}
\begin{itemize}
	
	\item \hypertarget{BEM}{Blade Element Moment Theory} - The theory used to calculate local forces on a propellor or wind turbine blade. It employs both \hyperlink{BET}{Blade Element Theory} and \hyperlink{MT}{Momentum Theory}. It calculates the 
	
	\item \hypertarget{BET}{Blade Element Theory} - Blade Element Theory calculates the forces on a turbine blade by dividing it into finite pieces and summing the forces on all of these pieces.
	
	\item \hypertarget{MT}{Momentum Theory} - Momentum Theory defines the power required to produce sufficient thrust to maintain momentum in a blade by the following equation, where $T$ is thrust, $\rho$ is density, $A$ is disc area, and $P$ is power:
	\begin{equation}
	\begin{aligned}
        		P = \sqrt{\frac{T^{3}}{2 \rho A}}
	\end{aligned}
	\end{equation}
	
	\item \hyperlink{phi}{Angle of Rotation, $\phi$} - The Angle of Rotation, sometimes denoted by the Greek letter $\phi$
	
\end{itemize}

\end{document}